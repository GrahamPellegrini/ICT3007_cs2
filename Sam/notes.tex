\documentclass{article}

\title{Case Study 2 - The Board Meeting - Sam}
\author{Graham Pellegrini}
\date{}

\begin{document}

\maketitle

\section{Comuters Sales Plan}
\subsection{a. The first 48 hours of running the business}
Within the first 48 hours of running the business, I am not planning to try make any major changes or statements. Rather, I see this time as an opportunity to observe the current state of the business from a managerial perspective. \\
- No major changes \\
- Observe the current state of the business \\

\begin{itemize}
    \item \textbf{Meeting with Steve} I would firstly like to have a meeting with Steve, so that he may provide me with further incentives on the current state of the business. As well as the revising the business vision and how the current actions taking place seek to follow this vision. \\
    \item \textbf{Meeting with the team} It is crutial to me that from the very beginning I establish a rapport with the team. I would like communication to be at the forefront of my management style. Allowing an ease of feedback and communication between myself and the team. \\
    \item \textbf{Begin Analyzes} Like I mentioned before, I would like to take the first 48 hours to observe the current state of the business. This would include reviewing the key metrics of the sales department, both customer and employee reviews and the current state of the market. \\
\end{itemize}
- Meeting with Steve; business vision and current actions \\
- Meeting with the team; establish rapport \\
- Begin analyzing the current state of the business \\

\subsection{b. The first week of running the business}
The first week of running the business is all about meeting and establishing rapport with key individuals inside and outside the business. \\

\begin{itemize}
    \item \textbf{Team Meetings} I would like to have a meeting with the department heads, so that I may understand the challenges they are facing and any possible ideas they may have. \\
    \item \textbf{Observe the day to day sales} I would like to observe the day to day sales that are taking place in store. This would allow me to see the current state of the business and how the team is currently operating. \\
    \item \textbf{Identify minor adjustments} I would like to identify any minor adjustments that could bring a positive impact to the business. However, I do not want to make too many minor changes quickly as their impact will still be felt. \\
\end{itemize}
- Team Meetings; understand challenges and ideas \\
- Observe the day to day sales \\
- Identify minor adjustments \\

\subsection{c. The first month of running the business}
The first month of running the business for me is about establishing short term goals and objectives. \\

\begin{itemize}
    \item \textbf{Short-term goals} I would define a measurable short-term goal that the team can work towards. This would allow me to see how the team operates and how they work towards a common goal. Identifying any possible issues that may arise, for bigger projects in which change is harder to implement. \\
    \item \textbf{Improvement Plan} I would like to create an improvement plan off of my observations. This would probably be aimed at driving enthusiasm to the team. With possible captial investment in training or resources where the team will benefit also on a personal level. \\
    \item \textbf{Meetings with Computer Engineering department} After settling more in a wish to begin meetings with the Computer Engineering department and Julie. This would allow departemental achievements to be celebrated and awknolaged not only by the board but also by the opposite department. Creating a sense of unity and teamwork between the two departments. \\
\end{itemize}
- Short-term goals; measurable goals \\
- Improvement Plan; drive enthusiasm \\
- Meetings with Computer Engineering department; unity and teamwork \\

\subsection{d. The first six months of running the business}
The first six months of running the business is where implementing long term goals and objectives come into play. \\

\begin{itemize}
    \item \textbf{Develop Talent:} Leveraging the rapport built with the team, I recognize the need for a small talent group within the division. This team would focus on rapidly implementing suggestions and ideas, enabling the business to respond more swiftly to market trends and changes. Members would be those with a keen personal interest in computers and technology, capable of identifying emerging products and trends. Possibly allowing us to bring in new products and services to the market faster than our competitors. \\
    \item \textbf{Strengthen customer relationships:} One of our key notes is the customer service we provide. I would like to ensure that we are providing the best customer service possible. This would include training for the team and possibly curate a new customer service strategy that would allow us to stand out from our competitors. \\
    \item \textbf{Streamline operations:} Contiue the work started in the first month of running the business. I would like to streamline the operations of the business. This would include identifying any inefficiencies and working towards a more efficient business model. \\
\end{itemize}
- Develop Talent; rapid implementation \\
- Strengthen customer relationships; best customer service \\
- Streamline operations; more efficient business model \\

\subsection{e. The first year of running the business}
The first year of running the business is about establishing the following years goals and objectives, as well as long term goals. In this time we also seek to keep the business as on of the market leaders in Malta. \\

\begin{itemize}
    \item \textbf{Avoid Stagnation:} I would like to avoid stagnation within the business. This would include keeping the team motivated and driven towards the goals and objectives that have been set. Through enagement such as the talent group and understanding the team's needs and wants. \\
    \item \textbf{Market Leader:} I would like to keep the business as one of the market leaders in Malta. This would include keeping up with the current market trends and changes.
    \item \textbf{Possible online store:} I would like to explore the possibility of an online store. This would allow us to reach a wider audience and possibly bring in more revenue. However, keeping in mind the challenges and risks that have kept us from doing so in the past. \\
    \item \textbf{Personal Mangering Development:} I would like to continue my personal development as a manager. I belive i already have the people skills needed to run the business, however, I would like to further develop my business acumen. \\
    \item \textbf{Meetings with the board:} I wish the board enviroment for the years to come to be one of openess and transparency. Like this egos and personal agendas can be put aside and the business can be run in the best possible way. \\
\end{itemize}
- Avoid Stagnation; keep the team motivated \\
- Market Leader; keep up with market trends \\
- Possible online store; reach a wider audience \\
- Personal Managering Development; further develop business acumen \\
- Meetings with the board; openess and transparency \\

\subsection{f.What leadership style should Sam employ?}
It is knowingly hard to be a leader and not a manager. However, I am determined to employ a democratic leadership style. This would allow me to involve the team in the decision-making process. I belive being able to trust the team and their decisions is a key part of being a leader. This would also allow me to keep the team motivated and driven towards the goals and objectives that have been set. I will obvoiusly still be the one to make the final decision and responsibility. However, I belive this invovement will allow the team to feel more valued and be a room of inspiring ideas. \\
- Democratic leadership style; involve the team in decision-making \\
- Keep the team motivated; driven towards goals and objectives \\
- Inspiring ideas; for business growth \\

Sam already has the people skills needed to run the business.

\subsection{g. How should he liaise with Julie and the Computer Engineering Division?}
So the two departements have very seperate operations. And their colobaration is rather trivial. If their is a disagreement between the departemental directions, the board meetings will be the only level of resolving this.

But i belive that through openess and transparency, Julie and I can put egos aside and look towards the interest of the business.

Aswell, as prevoiusly mentioned. Appart from the board meetings, I wish to have meetings with Julie and the respective department. These would take place whenever a commendable achievement has been made. This would allow the departments to celebrate and aknolage each other's achievements. Creating a sense of unity and teamwork between the two departments. \\

- Openess and transparency; put egos aside \\
- Meetings with Julie; celebrate achievements \\
- Unity and teamwork; between the two departments \\

\subsection{h. What are your first requests going to be to the board?}

\begin{itemize}
    \item \textbf{Capital Investment} I would like to request a capital investment in training or resources where the team will benefit also on a personal level. This would drive enthusiasm to the team and allow them to feel more valued. \\
    \item \textbf{Online Store} I would like to explore the possibility of an online store. This would allow us to reach a wider audience and possibly bring in more revenue. However, keeping in mind the challenges and risks that have kept us from doing so in the past. \\
    \item \textbf{Personal Development} I am open to the fact that I will make mistakes. So I wish to be to properly learn from them with guidance rather than solutions from Steve. \\
\end{itemize}

- Capital Investment; training or resources \\
- Online Store; reach a wider audience \\
- Personal Development; learn from mistakes \\

\section{The Software Bug}
\subsection{i. Warning Julie directly}
\textbf{Advantages:} \\
- The bug will likely be able to be fixed before the launch of the new product. \\
- It is an establishment of trust from Sam's side as he is willing to help Julie and show teamwork despite the departemental split. \\

\textbf{Disadvantages:} \\
- Julie may not take the warning well and may see it as a threat to her position. \\
- Sam may be seen as a snitch by the Computer Engineering Division. \\

\subsection{j. Not warning Julie or anyone}
\textbf{Advantages:} \\
- Sam avoids any confrontation with Julie by being a bystander. \\
- Sam may also see this as an opportunity to prove who is the better leader. \\

\textbf{Disadvantages:} \\
- The launch of the new product will likely flop and have terrible repercussions to the division that Julie runs. \\

\subsection{k. Warning the board of the problem in the next board meeting}
\textbf{Advantages:} \\
- The board will be able to make a decision on how to proceed with the launch of the new product. \\
- It shows that Sam is valuable to the board and is willing to help the business even outside of his departemental responsibilities. \\

\textbf{Disadvantages:} \\
- Julie may see this as a threat to her position. \\
- Damaging the relationship between Sam and Julie will lead to bigger problems between the two departments in the future as if the managers can't work together, how can the departments. \\
- The board may see this as a betrayal of trust between Sam and Julie. \\

\subsection{l. After considering the advantages and disadvantages, what do you think you should do?}
I belive that the best course of action would be to warn Julie directly.
Sam should appraoch Julie in a sincere and respectful manner. He should explain the situation and the possible repercussions of the software bug. This would allow Julie to take the necessary steps to fix the bug before the launch of the new product. \\

After the bug is fixed. In the next board meeting, Sam should bring up the situation and how it was resolved. Applauding Julie for her ability to fix the bug. And also shining light on his ability to see the bigger picture and help the business even outside of his departemental responsibilities. \\

\section{Minutes of the Board Meeting}

\end{document}
