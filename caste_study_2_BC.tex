\documentclass[a4paper,10pt]{article}
\usepackage{graphicx} % Required for in\textit{SER}ting images
\usepackage[a4paper,top=2cm,bottom=2cm,left=3cm,right=3cm,marginparwidth=1.75cm]
{geometry}
\usepackage[colorinlistoftodos]{todonotes}
\usepackage[style=apa, backend=biber]{biblatex}
%\usepackage{geometry}
\DeclareLanguageMapping{english}{english-apa}
\addbibresource{literature.bib} % Link your .bib file here

\title{ICT3007: Management of Computer Engineering Projects\newline \centering Case Study 1 - Succession Transition and Company Turnaround}

\author{
Bahne J. Thiel-Peters (bthi0001)
}

\begin{document}

\maketitle
\thispagestyle{empty}

\pagenumbering{arabic}
\setcounter{page}{1}

\section{Part C: Reflection}
\subsection{1.What did Julie decide to do about the leak of confidential information in Sam’s team? Discuss what she should have done? Why?}

What was done:
\begin{itemize}
    \item Julie decided to confront Sam directly about the leak of confidential information in his team.
    \item She decided not to escalate the issue to the board
    \item To avoid damaging Sams reputation and her own.
\end{itemize}

What should have been done and why:
\begin{itemize}
    \item Julie should have spoken to Sam in private about the leak of confidential information.
    \item She should have conviced and advised him to discuss this issue with Steve and John possibly at a board meeting.
    \item Since the leadership dynamic is small and still a family dynamic there shouldn't be reason to not discuss this issue with the board. 
    \item No ones reputation should be damaged if the issue is discussed in a professional mannerr.
\end{itemize}

\subsection{2. What did Sam decide to do about the software bug in Julie’s product? Discuss what he should have done? Why?}
\author{Graham Pellgrini}

What did Sam decide to do:
\begin{itemize}
    \item Sam decided to warn Julie directly
    \item He made sure to be sincere and respectful in his approach, making sure the situation was explained clearly.
    \item He would then later discuss the issue with the board, when the bug would be fixed and the time is right.
\end{itemize}

The group agreed with the actions taken by Sam. \\
By warning Julie directly, he made sure that she had the opportunity to fix the bug before it became a bigger issue. This act showed how Sam put the business first and is willing to work with Julie, showing good leadership skills.\\

He tries to enusre that Julie won't feel attacked by his confrontation, by being sincere and respectful. He then plans to discuss the issue with the board, when the bug is fixed and the time is right. This will shine light on both Julie for being able to handle the situation and fix the bug under pressure. As well as Sam for practicing the openess and transparency that he preached during the inital board meeting proposal.\\


\subsection{3. What did Steve decide to do about the conflict? Discuss what he should have done? Why?}

\subsection{4. What did John decide to do about monitoring Sam and Julie? Discuss what he should have done? Why?}

What did John do:
\begin{itemize}
    \item 
\end{itemize}

What should have been done and why:

\section{Post Announcement}
\subsection{5. Discuss the consequences of the announcement}
\author{Graham Pellgrini}
The announcement stated that John had created a fake scenario to test Sam and Julie.\\

\textbf{Positive Consequences:}
\begin{itemize}
    \item Through the announcement, Sam and Julie now know how each other would react in similar situations.
    \item This might bring them to trust each other further.
    \item John gained the reassurance that Sam and Julie are capable of working together and handling difficult situations.
\end{itemize}

\textbf{Negative Consequences:}
\begin{itemize}
    \item The announcement makes John seem unreliable, not respecting the family values that he preeched and enforced.
    \item Although it might have been intended for good reasons, it envoked a disturbing scene that they had to tackle during operation.
    \item Steve will see this action from John as unprofessional. There would have been other ways to test their leadership principles, without going behind their back and causing drama.
    \item Steve might be taken a back from Jonh's actions and might reconsider his position in the company.
    \item The announcement caused a rift between the family members, which is not desirable in a family business.
    \item An enviroment of uncertainty has been created, which puts other on edge.
    \item This issue could trickle down to the employees, causing them to feel uncertain about the company's future and the people managing it.
\end{itemize}

\subsection{6. Motivating the staff}

\begin{itemize}
    \item Coming up with revised mission and vission statements.
    \item Workload managemnt. We can see if we can distribute tasks better and schedule them better. (to address burnout)
    \item Development of a talent team, allowing particularly keen personnel to bring up new product ideas.
    \item Team building activites: dinners, sport events, etc.
    \item Improving the feedback system. (360 degree feedback)
    \item Speak to employees more often and check-in on them more frequently.
    \item Democratic leadirship style can help employees to feel like they have a voice within the company.
    \item Encourage Julie to involve her team more frequently by delegating important tasks so that the team can feel more involved.
\end{itemize}

\subsection{7. Role Changes and Responsibilities}

Option 1: Everyone keeps their current roles and responsibilities.\\

Advantages:\\
\begin{itemize}
    \item If things are going well, they likely will remain that way.
    \item No need to retrain or rehire.
    \item No need to change the company structure.
\end{itemize}

Disadvantages:\\
\begin{itemize}
    \item Family might not be comfortable with Steve remaining in such a powerful position.
    \item Steve likely wants to retire soon.
    \item Sam and Julie are ambitious and might want to take on more responsibilities.
\end{itemize}

Option 2: Steve remains on the Boards as a non-executive director. Sam becomes the CEO and Julie becomes the CTO.\\

Advantages:\\
\begin{itemize}
    \item Steve can still be involved in the company, so he can still use his experience to help the company, as well as Sam and Julie.
    \item Sam and Julie can take on more responsibilities.
    \item Steve can mentor Sam and Julie.
    \item Julie can use her technical knowledge to a better extent.
    \item Sam can use his leadership skills to a better extent.
    \item Julie can still help Sam with the technical strategy, but Sam now has enough experience and mentorship to make the final decisions.
\end{itemize}

Disadvantages:\\
\begin{itemize}
    \item Julie might be unhappy with the decision.
    \item Steve might not be able to let go of the company.
    \item Sam might not be ready to take on the role of CEO.
\end{itemize}

Option 3: Steve retires, and Sam and Julie joinlty take on the role of CEO.\\

Advantages:\\
\begin{itemize}
    \item Sam and Julie can work together to make decisions.
    \item Sam and Julie can use their skills to make the company more successful.
    \item Steve can retire and enjoy his retirement.'
    \item Sam and Julie can learn from eachother
\end{itemize}

Disadvantages:\\
\begin{itemize}
    \item Sam and Julie might not be able to work together.
    \item Steve might not be able to let go of the company.
    \item Disagreements might halt decision making, making the company less efficient.
\end{itemize}

We opted for the second option as this would allow Steve to still be involved in the company, while Sam and Julie can take on more responsibilities. Steve can mentor Sam and Julie, and Julie can use her technical knowledge to a better extent. Sam can use his leadership skills to a better extent. Julie can still help Sam with the technical strategy, but Sam now has enough experience and mentorship to make the final decisions.

\section{References}
\end{document}
