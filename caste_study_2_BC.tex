\documentclass[a4paper,10pt]{article}
\usepackage{graphicx} % Required for in\textit{SER}ting images
\usepackage[a4paper,top=2cm,bottom=2cm,left=3cm,right=3cm,marginparwidth=1.75cm]
{geometry}
\usepackage[colorinlistoftodos]{todonotes}
\usepackage[style=apa, backend=biber]{biblatex}
%\usepackage{geometry}
\DeclareLanguageMapping{english}{english-apa}
\addbibresource{literature.bib} % Link your .bib file here

\title{ICT3007: Management of Computer Engineering Projects\newline \centering Case Study 2 - The Board Meeting}

\author{
Phileas Barome\\
Graham Pellegrini\\
Luca Spiteri\\
Bahne J. Thiel-Peters\\
}

\begin{document}

\maketitle
\thispagestyle{empty}

\pagenumbering{arabic}
\setcounter{page}{1}

\section{Section B: Minutes}

\subsection{Sam's Minutes}
\author{Luca Spiteri (Steve)}

\subsubsection{Action Items}
\textbf{First 48 Hours}
- Meeting with Steve; business vision and current actions \\
- Meeting with the team; establish rapport \\
- Begin analyzing the current state of the business \\

\textbf{First Week}
- Team Meetings; understand challenges and ideas \\
- Observe the day to day sales physical in stores \\
- Identify minor adjustments, not too many changes \\

\textbf{First Month}
- Short-term goals; measurable goals \\
- Meetings with Computer Engineering department, celebrating sucesses to create unity \\

\textbf{First 6 Months}
- Develop Talent as proposed to Steve;
- Streamline operations on the already established system \\

\textbf{First Year}
- Self development in the business acumen as discussed with Steve \\
- Openness and transparency with board meetings \\


\subsubsection{Discussion}
\begin{itemize}
    \item Steve asked how Sam will balance consulting the team and making final decisions without delay.
    \item Sam plans to use a democratic leadership approach, involving the team in decision-making to drive enthusiasm and morale.
    \item Ultimately, Sam will make the final decision, considering the team's input.
    \item Julie inquired about maintaining ongoing communication.
    \item Sam's 6-month plan includes setting up a talent team to guide decisions with market insights.
    \item Steve questioned how Sam will maintain company culture.
    \item Sam believes his people skills align with the company culture and plans to develop his business acumen further.
    \item Sam and Julie discussed how they could handle the departemental meetings. Even delegating some weeks to departemental heads that could report back to them.
\end{itemize}

\subsubsection{Requests}
\begin{itemize}
    \item Funding for the talent team.
    \item Steve stated that this would be a investment worth looking into.
    \item Possiblilty to expand into online stores.
    \item John states that in this time of change it would be a good idea to look into this.
\end{itemize}

\subsection{John's Minutes}
\author{Graham Pellegrini (Sam)}

\subsubsection{Action Items}

\begin{itemize}
    \item \textbf{First 48 Hours:}
    \begin{itemize}
        \item Assess Julie and Sam's performance metrics for their respective divisions, focusing on immediate team feedback to address concerns about leadership styles.
        \item Request a summary of divisional goals from Steve, highlighting alignment with CompChimp's long-term strategic vision.
    \end{itemize}
    \item \textbf{First Week:}
    \begin{itemize}
        \item Convene a meeting with Julie to discuss her team's feedback and explore methods to incorporate team input into her decision-making process while maintaining agility.
        \item Hold a similar session with Sam to encourage more confident, independent decision-making while retaining his collaborative approach.
        \item Review Steve’s quarterly report outlining progress in mentoring Julie and Sam.
    \end{itemize}
    \item \textbf{First Month:}
    \begin{itemize}
        \item Implement anonymous surveys across both divisions to gauge employee morale, satisfaction, and engagement levels.
        \item Establish quarterly leadership workshops for Julie and Sam, focusing on transformational leadership and conflict resolution skills.
    \end{itemize}
    \item \textbf{Six Months:}
    \begin{itemize}
        \item Review division performance metrics, including sales growth, product innovation timelines, and employee retention rates.
        \item Conduct a mid-year evaluation of Steve’s mentorship impact on Julie and Sam’s readiness for increased responsibilities.
    \end{itemize}
    \item \textbf{One Year:}
    \begin{itemize}
        \item Present a detailed evaluation of Julie and Sam’s leadership development and performance at the annual board meeting.
        \item Facilitate discussions on the succession plan, including the possibility of promoting one of them to CEO while retaining mentorship from Steve.
    \end{itemize}
\end{itemize}

\subsubsection{Discussion}

\begin{itemize}
        \item Steve acknowledged Julie’s decisiveness but emphasized the need for her to engage her team more to maintain morale and encourage innovation.
        \item He suggested that Sam focus on setting clear boundaries for consultation and decision-making to strengthen his leadership presence.
    \end{itemize}
    \begin{itemize}
        \item Sam highlighted that Julie’s directive style might lead to long-term resistance among her team members, which could affect productivity.
        \item He proposed that she consider delegating decision-making authority in areas where team members excel.
    \end{itemize}
    \begin{itemize}
        \item Julie pointed out that Sam’s consultative style could delay decisions critical to sales momentum. She encouraged him to balance input with timely action.
        \item She recommended that he practice making smaller, independent decisions to build confidence.
    \end{itemize}
    \begin{itemize}
        \item Introduce a peer mentorship system between Sam and Julie, allowing them to learn from each other’s strengths and address weaknesses collaboratively.
        \item Assign trial projects requiring both decisive action and team collaboration to refine their leadership approaches.
    \end{itemize}

\subsubsection{Request}

\begin{itemize}
    \item Approve funding for quarterly leadership workshops tailored to developing Julie and Sam’s transformational leadership capabilities.
    \item Allocate resources for anonymous team surveys to track morale and engagement consistently.
    \item Extend Steve’s mentorship role for an additional year to guide Julie and Sam through their leadership transition, with a performance review at month 18 to decide on future senior management structures.
\end{itemize}


\subsection{Steve's Minutes}
\author{Bahne Thiel-Peters (Julie)}

\subsubsection{Action Items}

\noindent \textbf{First 48 Hours: }
\begin{itemize}
    \item Familiarize with the company culture and structure.
    \item Meet key personnel and establish rapport.
    \item Identify immediate operational issues requiring attention.
\end{itemize}

\noindent \textbf{First Week:}
\begin{itemize}
    \item Familiarize with the company culture and structure.
    \item Meet key personnel and establish rapport.
    \item Identify immediate operational issues requiring attention.
\end{itemize}

\noindent \textbf{First Month:}
\begin{itemize}
    \item Implement a formal review system to track divisional progress.
    \item Introduce any necessary structural adjustments.
    \item Conduct regular meetings with Sam and Julie to clarify responsibilities and foster open communication.
\end{itemize}

\noindent \textbf{First Six Months}
\begin{itemize}
    \item Set measurable milestones for Sam and Julie with a focus on outcomes.
    \item Facilitate team-building activities to encourage collaboration across departments.
    \item Closely monitor performance and provide guidance to align divisional efforts with company objectives.
\end{itemize}

\noindent \textbf{First Year}
\begin{itemize}
    \item Review progress toward strategic objectives and make necessary adjustments.
    \item Conduct a formal evaluation of Sam and Julie’s leadership, incorporating feedback from their teams.
\end{itemize}

\noindent \textbf{First Two Years}
\begin{itemize}
    \item Ensure divisions have sustainable growth strategies and clearly defined goals.
    \item Introduce initiatives to reward cross-departmental collaboration, supporting long-term stability.
\end{itemize}

\subsubsection{Discussion}
\noindent \textbf{Steve's Plan}
\begin{itemize}
    \item Emphasis on building rapport and understanding team dynamics.
    \item Long-term focus on sustainability and collaboration across divisions.
    \item Monitoring strategy praised for balance and structure.
\end{itemize}

\noindent \textbf{Critiques}
\begin{itemize}
    \item John Expressed concern over potential delays in immediate outcomes due to mediation-centric approaches.
    \item Julie suggested considering additional metrics in the scorecard for dynamic business environments.
\end{itemize}

\subsubsection{Requests}
\noindent \textbf{From Steve:}
\begin{itemize}
    \item Approval to introduce structured team-building initiatives within the first six months.
    \item Feedback on the proposed balanced scorecard metrics and inclusion of qualitative KPIs.
\end{itemize}

\noindent \textbf{From Sam and Julie:}
\begin{itemize}
    \item Agreement on clear boundaries for resource use, including meeting room allocation.
    \item Support for resolving overlapping operational areas to reduce tension.
\end{itemize}

\noindent \textbf{From John:}
\begin{itemize}
    \item Comprehensive evaluation of Sam and Julie at month 12 to determine leadership capabilities.
    \item Proposal for a mid-year check-in to assess Steve’s performance as CEO.
\end{itemize}

\subsection{Julie's Minutes}
\author{Phileas Barome (John)}

\subsubsection{Action Items}

\noindent \textbf{First 48 Hours}
\begin{itemize}
    \item Meet Key Team Members.
    \item Review Current Projects and Finances.
    \item Identify Immediate Risks and Opportunities
\end{itemize}

\noindent \textbf{First Week}
\begin{itemize}
    \item Introduce \textit{weekly standup meetings} to get to know peoples strengths and characteristics.
    \item Set team expectations and define goals.
\end{itemize}

\noindent \textbf{First Month}
\begin{itemize}
    \item Address initial inefficiencies discovered in the first week.
    \item Draft a basic roadmap for expanding the engineering division’s offerings.
    \item Work closely with Sam’s team in Computer Sales to identify areas where the Engineering Division can support increased sales or product offerings.
    \item Build Team Morale by starting events after the work or on weekends.

\end{itemize}

\noindent \textbf{First Six Months}
\begin{itemize}
    \item Launch a New Tech Innovation Hub within the Division, to introduce new technologies.
    \item Rethink and eventually refine Division Structure and Roles.
    \item Monitor Performance Closely. What's with the previously made goals from the last months?
\end{itemize}

\noindent \textbf{First Year}
\begin{itemize}
    \item Review steps from the first half year.
    \item Solidify Division Strategy and Vision: Develop a long-term strategy for the Computer Engineering Division.
    \item Implement Key Structural Changes: Formalize and optimize the organizational structure within the division - clear reporting lines and job descriptions. 
\end{itemize}

\subsubsection{Discussion}

\begin{itemize}
    \item Sam criticized Julie's leadership, highlighting her challenges with social interactions as a potential obstacle to effective team management.
    \item John stressed the need for Julie and Sam to collaborate to align their divisions.
\end{itemize}

\subsubsection{Requests}

\begin{itemize}
    \item Request \textit{funding for research and development}, enabling the Engineering Division to stay competitive and expand into new areas. 
    \item Request \textit{support for training programs} or hiring additional staff with specialized skills.
    \item Request board approval for organizational changes to support team growth.
\end{itemize}

\section{Part C: Reflection}
\subsection{1.What did Julie decide to do about the leak of confidential information in Sam’s team? Discuss what she should have done? Why?}

What was done:
\begin{itemize}
    \item Julie decided to confront Sam directly about the leak of confidential information in his team.
    \item She decided not to escalate the issue to the board
    \item To avoid damaging Sams reputation and her own.
\end{itemize}

What should have been done and why:
\begin{itemize}
    \item Julie should have spoken to Sam in private about the leak of confidential information.
    \item She should have conviced and advised him to discuss this issue with Steve and John possibly at a board meeting.
    \item Since the leadership dynamic is small and still a family dynamic there shouldn't be reason to not discuss this issue with the board. 
    \item No ones reputation should be damaged if the issue is discussed in a professional mannerr.
\end{itemize}

\subsection{2. What did Sam decide to do about the software bug in Julie’s product? Discuss what he should have done? Why?}

What did Sam decide to do:
\begin{itemize}
    \item Sam decided to warn Julie directly
    \item He made sure to be sincere and respectful in his approach, making sure the situation was explained clearly.
    \item He would then later discuss the issue with the board, when the bug would be fixed and the time is right.
\end{itemize}

The group agreed with the actions taken by Sam. \\
By warning Julie directly, he made sure that she had the opportunity to fix the bug before it became a bigger issue. This act showed how Sam put the business first and is willing to work with Julie, showing good leadership skills.\\

He tries to enusre that Julie won't feel attacked by his confrontation, by being sincere and respectful. He then plans to discuss the issue with the board, when the bug is fixed and the time is right. This will shine light on both Julie for being able to handle the situation and fix the bug under pressure. As well as Sam for practicing the openess and transparency that he preached during the inital board meeting proposal.\\


\subsection{3. What did Steve decide to do about the conflict? Discuss what he should have done? Why?}

\subsubsection{What did Steve decide to do:}
Steve took 3 steps to mediate the conflict between Sam and Julie:

\begin{itemize}
    \item He facilitated a structured discussion to address the grievances of both parties.
    \item He allocated limited resources to Julie's RnD efforts to support her innovation while ensuring Sam's sales campaigns remain operational.
    \item He developed a balanced resource-sharing policy to clarify and enforce boundaries for office space, meeting rooms, and personnel allocation.
\end{itemize}

\subsubsection{What should have been done and why:}

While the steps taken by Steve were reasonable, he could have considerred taking some additional steps to address the root cause of the conflict with more effect.\\
\begin{itemize}
    \item Establish Clear Role Definitions from the Outset, in order to clearly define Sam and Julie's boundaries of authority and responsibility.
    \item Create a Joint Planning Framework, so that Sam and Julie can work together to develop a shared vision and strategy for the company.
    \item Address the Underlying Cultural Issue, there might have been competitive pressures, team building activities and meetings could have been used to address this.
\end{itemize}

These steps are important to include as they help to prevent the issue from happening again, rather than simply solving it when it cropped up.

\subsection{4. What did John decide to do about monitoring Sam and Julie? Discuss what he should have done? Why?}

\subsubsection{What did John decide to do:}
John implemented several measures to monitor Sam and Julie's progress and performance:

\begin{itemize}
    \item He evaluated their performance using quantitative metrics such as sales growth, market penetration, and timely delivery of new products.
    \item He collected qualitative feedback through periodic one-on-one sessions with Sam and Julie, as well as by observing their interactions during board meetings.
    \item He occasionally gathered team feedback to assess their leadership styles and the morale of their respective divisions.
    \item He reviewed quarterly divisional reports to ensure both leaders were meeting their set goals and expectations.
\end{itemize}

\subsubsection{What should have been done and why:}
Although John's measures were a step in the right direction, a more comprehensive and proactive approach to monitoring could have been implemented to address leadership and team challenges more effectively. This could have included the following:

\begin{itemize}
    \item \textbf{Establish Regular Employee Surveys:} Conducting anonymous surveys to monitor team morale, workload, and satisfaction on a regular basis. This would provide insights into potential burnout and dissatisfaction before they escalate.
    \item \textbf{Introduce Real-Time Feedback Mechanisms:} Encouraging open communication between Sam, Julie, and their teams, creating a culture where feedback is provided and acted upon continuously.
    \item \textbf{Define Key Leadership KPIs:} Setting clear, measurable leadership KPIs focused not only on divisional performance but also on employee retention, team satisfaction, and leadership effectiveness.
    \item \textbf{Provide Mentorship and Leadership Training:} Establishing regular mentorship sessions for Sam and Julie to address their leadership gaps—helping Sam become more decisive and Julie more inclusive.
    \item \textbf{Monitor Workload and Resource Allocation:} Regularly reviewing team workloads to ensure equitable distribution and prevent burnout, ensuring teams are not stretched beyond their capacity.
\end{itemize}


\section{Post Announcement}
\subsection{5. Discuss the consequences of the announcement}
The announcement stated that John had created a fake scenario to test Sam and Julie.\\

\textbf{Positive Consequences:}
\begin{itemize}
    \item Through the announcement, Sam and Julie now know how each other would react in similar situations.
    \item This might bring them to trust each other further.
    \item John gained the reassurance that Sam and Julie are capable of working together and handling difficult situations.
\end{itemize}

\textbf{Negative Consequences:}
\begin{itemize}
    \item The announcement makes John seem unreliable, not respecting the family values that he preeched and enforced.
    \item Although it might have been intended for good reasons, it envoked a disturbing scene that they had to tackle during operation.
    \item Steve will see this action from John as unprofessional. There would have been other ways to test their leadership principles, without going behind their back and causing drama.
    \item Steve might be taken a back from Jonh's actions and might reconsider his position in the company.
    \item The announcement caused a rift between the family members, which is not desirable in a family business.
    \item An enviroment of uncertainty has been created, which puts other on edge.
    \item This issue could trickle down to the employees, causing them to feel uncertain about the company's future and the people managing it.
\end{itemize}

\subsection{6. Motivating the staff}

\begin{itemize}
    \item Revise the company’s mission and vision statements to inspire and re-align the team with organizational goals.
    \item Implement effective workload management to better distribute and schedule tasks, reducing burnout.
    \item Establish a \textit{talent development team} to engage employees in proposing innovative product ideas.
    \item Organize \textit{team-building activities}, such as dinners, sports events, and other social engagements, to foster camaraderie.
    \item Enhance the \textit{feedback system} by introducing 360-degree feedback to provide comprehensive and constructive performance insights.
    \item Increase communication through frequent check-ins with employees to address concerns and improve morale.
    \item Adopt a \textit{democratic leadership style} to ensure employees feel heard and valued within the organization.
    \item Encourage Julie to delegate important tasks to her team, promoting involvement and ownership in the decision-making process.
\end{itemize}

\textbf{Why care about retaining the staff?}\newline
Sam and Julie should prioritize retaining their staff to ensure continuity, maintain productivity, and avoid the high costs of recruitment and training. Experienced employees hold valuable knowledge that drives innovation and sustains growth, while frequent turnover disrupts morale and team efficiency. Retaining motivated staff fosters a positive reputation, attracts talent, and supports long-term success by keeping the company competitive and aligned with strategic goals.

\subsection{7. Role Changes and Responsibilities}

The company faces a critical decision regarding its future leadership structure. Three potential scenarios are being considered, each with its own implications. Below, the advantages and disadvantages of each option are explored in detail.

\subsubsection*{Option 1: Maintaining the Current Roles and Responsibilities}

\subsubsection*{Advantages}
This approach ensures continuity, which can be valuable if the company is already performing well. The existing structure avoids disruption, as there would be no need to retrain or hire new personnel. Furthermore, the current organizational framework would remain intact, simplifying operations.

\subsubsection*{Disadvantages}
While maintaining the status quo provides stability, it may not address the ambitions of Sam and Julie, who are eager for greater responsibilities. Additionally, Steve’s prolonged role as CEO could lead to discomfort among family members, as he continues to wield significant influence. Steve may also be nearing retirement, and the absence of a transition plan could leave the company unprepared for his eventual departure.

\subsubsection*{Option 2: Transitioning Steve to a Non-Executive Director Role, with Sam as CEO and Julie as CTO}

\subsubsection*{Advantages}
This structure allows Steve to step back from daily operations while continuing to provide guidance as a non-executive director. Both Sam and Julie would have the opportunity to take on larger roles, aligning with their ambitions. Sam’s leadership skills could be more effectively utilized in the CEO role, while Julie’s technical expertise would thrive in the CTO position. Steve’s mentoring role ensures continuity, and Julie could still contribute to technical strategy while Sam gains the confidence and experience needed to make final decisions.

\subsubsection*{Disadvantages}
There is a risk that Julie may feel overlooked if she aspires to the CEO position, which could impact her satisfaction and motivation. Additionally, Steve might find it difficult to fully let go of operational control, which could create tension during the transition. Lastly, while Sam has shown leadership potential, he may still lack the experience needed to handle the full scope of the CEO responsibilities.

\subsubsection*{Option 3: Joint Leadership by Sam and Julie as Co-CEOs, with Steve Retiring}

\subsubsection*{Advantages}
This option allows Sam and Julie to share decision-making responsibilities, leveraging their combined strengths to lead the company effectively. By working together, they can learn from one another, balancing Sam’s leadership style with Julie’s technical expertise. Additionally, Steve’s retirement would give him the opportunity to enjoy personal time while ensuring the family maintains control of the company.

\subsubsection*{Disadvantages}
Shared leadership poses risks, as disagreements between Sam and Julie could delay decision-making and affect the company’s efficiency. There is also a chance that their working relationship could become strained under the pressures of joint leadership. Furthermore, Steve may struggle to completely detach from the company, potentially complicating the transition.

\subsubsection*{Decision and Justification}

After evaluating all options, the second option is determined to be the most suitable. This approach strikes a balance between progression and stability. Steve’s transition to a non-executive director role ensures his experience and guidance remain available. Sam’s appointment as CEO allows him to refine his leadership skills, while Julie’s new role as CTO leverages her technical expertise to drive innovation. This structure also fosters collaboration, as Julie continues to support Sam in technical strategy while he gains confidence in his decision-making. Overall, this arrangement prepares the company for sustainable growth while maintaining alignment with family values.

\section*{Bibliography}\nocite{*}

\begin{itemize}
    \item OpenAI. (2024). \textit{ChatGPT: Language Model by OpenAI.} \\ 
    This tool was used to formalise bullet points to sentences.
\end{itemize}

\pagebreak

\end{document}
